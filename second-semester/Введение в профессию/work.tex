\documentclass[a4paper,12pt,onecolumn]{article}
\usepackage{amsmath}
\usepackage[T2A]{fontenc} 
\usepackage{cmap} 
\usepackage[russian]{babel} 
\usepackage{textcomp} 
\usepackage{indentfirst} 
\usepackage{amssymb}
\usepackage{amsmath} 
\usepackage{graphicx} 
\usepackage{geometry} 
\geometry{a4paper, total={170mm,257mm}, left=20mm, top=10mm, bottom=20mm, right=20mm}
\begin{document}
\title{Заголовок}
\author{Ердяков Р.А.}
\date{2024.06.26}
\maketitle

$$ \delta_{ij} =
\begin{cases}
1, & i=j,\\
0, & i\ne j.
\end{cases} $$
\newline
Текст % А
будет не % комментарий
прерывен! % пропадет!
% Рисование
\newline
\rule[10pt]{25pt}{1pt}
\newline
\rule[-5pt]{10pt}{10pt}
\newline
\rule{6pt}{6pt}
\newline
\begin{itshape}
Для верстки большого объема текста лучше пользовать окружениями
\end{itshape}
\newline
\appendix
\section{Название}\label{section_name}
Текст приложения\dots
\subsection{Заголовок 1}\label{section_1}
Продолжение\dots
\subsection{Заголовок 2}\label{section_2}
\tableofcontents

Номер станицы с меткой название - \pageref{section_name}

\setcounter{footnote}{0}
\addtocounter{footnote}{2}
Текущее значение счетчика footnote = \thefootnote

\footnote{Текст в примечении}{Текст который будет со сноской}
\end{document}
